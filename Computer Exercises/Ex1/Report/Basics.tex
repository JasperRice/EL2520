\section{Basics}
\image{0.75}{system}{$F$-controller, $G$-system, $r$-reference signal, $u$-control signal, $d$-disturbance signal, $y$-output signal, $n$-measurement noise.}

\par Consider a system which can be modeled by the transfer function
\begin{align*}
	G(s) &= \frac{3(-s+1)}{(5s+1)(10s+1)}
\end{align*}
\begin{enumerate}
	\item \textbf{Question:} Use the procedure introduced in the basic course to construct a lead-lag controller which eliminates the static control error for a step response in the reference signal.
	\begin{align*}
		F(s) &= K\underbrace{\frac{\tau_{D}s+1}{\beta\tau_{D}s+1}}_{\mathrm{Lead}}\underbrace{\frac{\tau_{I}s+1}{\tau_{I}s+\gamma}}_{\mathrm{Lag}}
	\end{align*}
	\par The phase margin should be $30\degree$ at the cross-over frequency $\omega_{c}=0.4$ rad/s.
	\par \textbf{Answer:} For the closed-loop system, in order to have a phase margin $\phi_{m}=30\degree$, a cross-over frequency $\omega_{c}=0.4$ rad/s, and zero steady-state error for a step response in the reference signal, we consider a lead-lag controller $F$ shown above, where $K$, $\tau_{D}$, $\tau_{I}$, $\beta$, and $\gamma$ are parameters should be configured so that the closed-loop system satisfies the requirements.
	\par For a step reference, the error is given by
	\begin{align*}
		E(s) = R(s) - Y(s) = \frac{1}{1+F(s)G(s)}R(s) = \frac{1}{1+F(s)G(s)}\frac{1}{s}
	\end{align*}
	\par The steady-state error then can be obtained using Final Value Theorem:
	\begin{align*}
		e(\infty) = \lim_{t\rightarrow 0}e(t) = \lim_{s\rightarrow 0} sE(s) = \frac{1}{1+F(0)G(0)} = \frac{\gamma}{\gamma + 3K}
	\end{align*}
	\par To get zero steady-state error, $e(\infty) = 0$, either $\gamma = 0$ or $K \rightarrow \infty$. So, $\gamma = 0$ is chosen here.
	\par Then we can design the lag controller. To minimize the phase lag caused by the lag compensator, we would obtain:
	\begin{align*}
		\tau_{I} = \frac{10}{\omega_{c}} = \frac{10}{0.4} = 25
	\end{align*}
	then the lag component becomes to $F_{I}(s) = \frac{25s+1}{25s}$, and the phase margin of $F_{I}(s)G(s)$ is equal to 
	
	\par In all, we have our controller:
	\begin{align*}
		F(s) = 
	\end{align*}
	
	\item \textbf{Question:} 
	\par \textbf{Answer:}
	
	\item \textbf{Question:} 
	\par \textbf{Answer:} 
\end{enumerate}