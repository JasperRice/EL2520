\documentclass[a4paper,11pt]{article}
\usepackage[utf8]{inputenc}
\usepackage[T1]{fontenc}
\usepackage[english]{babel}
\usepackage{amsmath,amssymb,amsthm,amsfonts}
\usepackage[demo]{graphicx} % REMOVE "demo" to include figures!

\title{
	Computer Exercise 1\\
	EL2520 Control Theory and Practice
}
\author{
	Sifan Jiang\\
	sifanj@kth.se\\
	961220-8232
	\and
	Oscar Oscarsson\\
	y@kth.se\\
	YYMMDD-NNNN
}

\newcommand{\image}[3][width=1.0\columnwidth]{
	\begin{figure}[!ht]
		\centering
	    \includegraphics[#1]{#2}
		\caption{#3}
		\label{fig:#2}
	\end{figure}
}

\begin{document}
	\maketitle

	% Minimum phase case
	\section*{Disturbance attenuation}
	
	How should the extra poles be chosen in exercise 4.2.1? Motivate! 
	\par\dotfill\par\dotfill\par

	The feedback controller in exercise 4.2.2 is
	\[
		F_y(s) = \ldots
	\]

	\image{figure_1.pdf}{Step disturbance, exercise 4.2.2}
	
	The feedback controller and prefilter in exercise 4.2.3 is 
	\[
	F_y(s) = \ldots
	\]
	\[
	F_r(s) = \ldots
	\]
	\image{figure_2.pdf}{Reference step, exercise 4.2.3}
	\image{figure_3.pdf}{Control signal for a disturbance or a reference step (plus a combination of these)}

	Did you manage to fulfill all the specifications? If not, what do you think makes the specifications difficult to achieve?
	\par\dotfill\par\dotfill\par
	
	\image{figure_4.pdf}{Bode diagram of sensitivity and complementary sensitivity functions, exercise 4.2.4}

\end{document}
