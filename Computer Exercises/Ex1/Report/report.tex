\documentclass[11pt,a4paper]{article}
\usepackage[utf8]{inputenc}
\usepackage[T1]{fontenc}
\usepackage[english]{babel}

\title{
	Computer Exercise 1\\
	EL2520 Control Theory and Practice
}
\author{
	Sifan Jiang\\
	sifanj@kth.se\\
	961220-8232
	\and
	Jiaqi Li\\
	jiaqli@kth.se\\
	960326-1711
}

\newcommand{\image}[3]{
	\begin{figure}[!ht]
		\centering
	    \includegraphics[width=#1\textwidth]{#2}
		\caption{#3}
		\label{fig:#2}
	\end{figure}
}

% My packages
\usepackage{algorithm, algorithmic, listings} % Code
\usepackage{amsmath, amstext, amssymb, amsfonts, amsthm, dsfont, cancel, gensymb, mathtools, textcomp} % Math
\usepackage{color, xcolor} % Color
\usepackage{diagbox, tabularx} % Table
\usepackage{enumerate} % List
\usepackage{epsfig, epstopdf, graphicx, multicol, multirow, palatino, pgfplots, subcaption, tikz} % Image.
\usepackage{fancybox}
\usepackage{verbatim}

% \usepackage[font=footnotesize]{caption} % labelfont=bf
% \usepackage[font=scriptsize]{subcaption} % labelfont=bf
\usepackage[margin=1in]{geometry}
\usepackage[hidelinks]{hyperref}
\epstopdfsetup{outdir=./Figure/Converted/}
\graphicspath{{./Figure/}}

\makeatletter
\def\input@path{{./Figure/}}
\makeatother

\pgfplotsset{compat=1.13}

\begin{document}
\maketitle

\section{Exercises}
\subsection{Basics}
\image{0.75}{system}{$F$-controller, $G$-system, $r$-reference signal, $u$-control signal, $d$-disturbance signal, $y$-output signal, $n$-measurement noise.}

\par Consider a system which can be modeled by the transfer function
\begin{align*}
	G(s) &= \frac{3(-s+1)}{(5s+1)(10s+1)}
\end{align*}
\begin{enumerate}
	\item \textbf{Question:} Use the procedure introduced in the basic course to construct a lead-lag controller which eliminates the static control error for a step response in the reference signal.
	\begin{align*}
		F(s) &= K\underbrace{\frac{\tau_{D}s+1}{\beta\tau_{D}s+1}}_{\mathrm{Lead}}\underbrace{\frac{\tau_{I}s+1}{\tau_{I}s+\gamma}}_{\mathrm{Lag}}
	\end{align*}
	\par The phase margin should be $30\degree$ at the cross-over frequency $\omega_{c}=0.4$ rad/s.
	\par \textbf{Answer:} For the closed-loop system, in order to have a phase margin $\phi_{m}=30\degree$, a cross-over frequency $\omega_{c}=0.4$ rad/s, and zero steady-state error for a step response in the reference signal, we consider a lead-lag controller $F$ shown above, where $K$, $\tau_{D}$, $\tau_{I}$, $\beta$, and $\gamma$ are parameters should be configured so that the closed-loop system satisfies the requirements.
	\par For a step reference, the error is given by
	\begin{align*}
		E(s) = R(s) - Y(s) = \frac{1}{1+F(s)G(s)}R(s) = \frac{1}{1+F(s)G(s)}\frac{1}{s}
	\end{align*}
	\par The steady-state error then can be obtained using Final Value Theorem:
	\begin{align*}
		e(\infty) = \lim_{t\rightarrow 0}e(t) = \lim_{s\rightarrow 0} sE(s) = \frac{1}{1+F(0)G(0)} = \frac{\gamma}{\gamma + 3K}
	\end{align*}
	\par To get zero steady-state error, $e(\infty) = 0$, either $\gamma = 0$ or $K \rightarrow \infty$. So, $\gamma = 0$ is chosen here.
	\par Then we can design the lag controller. To minimize the phase lag caused by the lag compensator, we would obtain:
	\begin{align*}
		\tau_{I} = \frac{10}{\omega_{c}} = \frac{10}{0.4} = 25
	\end{align*}
	then the lag component becomes to $F_{I}(s) = \frac{25s+1}{25s}$, and the phase margin of $F_{I}(s)G(s)$ is equal to 
	
	\par In all, we have our controller:
	\begin{align*}
		F(s) = 
	\end{align*}
	
	\item \textbf{Question:} 
	\par \textbf{Answer:}
	
	\item \textbf{Question:} 
	\par \textbf{Answer:} 
\end{enumerate}


\subsection{Disturbance attenuation}
\image{0.75}{system2}{$F_{r}$-prefilter, $F_{y}$-feedback controller, $G$-system, $G_{d}$-disturbance dynamics, $r$-reference signal, $u$-control signal, $d$-disturbance signal, $y$-measurement signal.}

\par The block diagram of the control system is given in fig~\ref{fig:system2}, where the transfer functions have been estimated to
\begin{align*}
	G(s) &= \frac{20}{(s+1)((\frac{s}{20})^{2}+\frac{s}{20}+1)} \\
	G_{d}(s) &= \frac{10}{s+1}
\end{align*}

\begin{enumerate}
	\item \textbf{Question:} For which frequencies is control action needed? Control is needed at least at frequencies where $\vert G_{d}(j\omega) \vert > 1$ in order for disturbances to be attenuated. Therefore the cross-over frequency must be large enough. First, try to design $F_{y}$ such that $L(s) \approx \omega_{c} / s$ and plot the closed-loop transfer function from $d$ to $y$ and the corresponding step response. (A simple way to find $L(s) = \omega_{c} / s$ is to let $F_{y} = G^{-1} \omega_{c} / s$. However, this controller is not proper. A procedure to fix this is to ``add'' a number of poles in the controller such that it becomes proper. How should these poles be chosen?)
	\par \textbf{Answer:} $\omega_{c} = 9.9473$ rad/s is the cross-over frequency of $G_{d}$ so that $G_{d}(j\omega_{c}) = 1$. And since control is needed at least at frequencies where $\vert G_{d}(j\omega) \vert > 1$, $\omega \in [0, 9.9473]$ rad/s is the frequencies which control action needed.
	\par To design $F_{y}$ such that $L_{s} \approx \omega_{c} / s$, a naive approach can be first considered:
	\begin{align*}
		F_{y}^{*}(s) &= G^{-1}(s) \omega_{c} / s \\
		&= \frac{(s+1)((\frac{s}{20})^{2}+\frac{s}{20}+1)\omega_{c}}{20s}
	\end{align*}
	so that $L^{*}(s) = F_{y}^{*}(s) G(s) = \omega_{c} / s$. However, the relative degree of $F_{y}^{*}$ is $-2$, meaning future information in time domain would be needed, which is impossible. So poles such be ``added'' to make the controller proper.
	\par To make the relative degree to be at least $0$, two poles, $p_{1}$ and $p_{2}$, are added such that
	\begin{align*}
		F_{y}(s) &= \frac{G^{-1}(s)\omega_{c}p_{1}p_{2}}{s(s+p_{1})(s+p_{2})} \\
		&= \frac{(s+1)((\frac{s}{20})^{2}+\frac{s}{20}+1)\omega_{c}}{20s} \cdot \frac{p_{1}p_{2}}{(s+p_{1})(s+p_{2})}
	\end{align*}
	To simplify the addition of poles, poles are set such that $p_{1}=p_{2}=p$.
	\par The open-loop transfer function becomes to $L(s) = \frac{p_{1}p_{2}\omega_{c}}{s(s+p_{1})(s+p_{2})}$. So the closed-loop transfer function from $d$ to $y$ is $G_{d_{cl}}=\frac{G_{d}}{1+L}$. The characteristics equation of $G_{d_{cl}}$ is
	\begin{align*}
		s^{4} + (2p+1)s^{3} + (2p+p^{2})s^{2} + p^{2}(w_{c}+1)s + p^{2}w_{c} = 0
	\end{align*}
	\par To make the system stable, the poles in the characteristics equation should be in the LHP. To satisfy such requirement, Routh-Hurwitz Stability Criterion is used to test the stability and find possible value for $p$. Thus, a sufficient condition is that $p$ should be positive and large enough. However if the value of poles added is too large, the step response would be too rapid and the amplitude of step response would also be too large. $p=100$ is finally chosen as a decision of the trade-off. The bode plot of the open-loop
\end{enumerate}

\par The feedback controller in exercise 4.2.2 is
\begin{align*}
	F_{y}(s) &= \ldots
\end{align*}

	\image{0.5}{system}{Step disturbance, exercise 4.2.2}
	
	The feedback controller and prefilter in exercise 4.2.3 is 
	\[
	F_y(s) = \ldots
	\]
	\[
	F_r(s) = \ldots
	\]
	\image{0.25}{system}{Reference step, exercise 4.2.3}
	\image{0.25}{system}{Control signal for a disturbance or a reference step (plus a combination of these)}

	Did you manage to fulfill all the specifications? If not, what do you think makes the specifications difficult to achieve?
	\par\dotfill\par\dotfill\par
	
	\image{0.25}{system}{Bode diagram of sensitivity and complementary sensitivity functions, exercise 4.2.4}

\end{document}
